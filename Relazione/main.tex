%%%%%%%%%%%%%%%%%%%%%%%%%%%%%%%%%%%%%%%%%%%%%%%%%%%%%%%%%%%%%%%%%%%%%%%%%%%%%%%%%%%%%%%%%%%%%%%%%%%%%%%%
%Paolo Eccher, Andrea Favero, Cristian Maschio, Francesco Parolini
%%%%%%%%%%%%%%%%%%%%%%%%%%%%%%%%%%%%%%%%%%%%%%%%%%%%%%%%%%%%%%%%%%%%%%%%%%%%%%%%%%%%%%%%%%%%%%%%%%%%%%%%

\documentclass[10pt, a4paper]{article}

\usepackage[scaled]{helvet}

\usepackage[utf8]{inputenc}
\usepackage[T1]{fontenc}
\usepackage[italian]{babel}

\usepackage{graphicx}
\usepackage{fix-cm}
\newcommand{\bigsize}{\fontsize{35pt}{20pt}\selectfont}
\newcommand{\mediumsize}{\fontsize{30pt}{20pt}\selectfont}
\newcommand{\normsize}{\fontsize{15pt}{10pt}\selectfont}

\usepackage{float}
\usepackage{caption}


\usepackage{multirow}
%crea una cella per le tabelle in grado di andare a capo con \newline
%https://tex.stackexchange.com/questions/12703/how-to-create-fixed-width-table-columns-with-text-raggedright-centered-raggedlef
\usepackage{array}
\newcolumntype{L}[1]{>{\raggedright\let\newline\\\arraybackslash\hspace{0pt}}m{#1}}
\newcolumntype{C}[1]{>{\centering\let\newline\\\arraybackslash\hspace{0pt}}m{#1}}
\newcolumntype{R}[1]{>{\raggedleft\let\newline\\\arraybackslash\hspace{0pt}}m{#1}}

%https://tex.stackexchange.com/questions/4503/how-do-i-specify-color-in-rgb-using-hypersetup-in-hyperref
\usepackage{url}
\usepackage{breakurl}
\usepackage[colorlinks=true]{hyperref}
\usepackage[hyperref]{xcolor}
\definecolor{UniPD}{RGB}{155, 0, 20}
\hypersetup{colorlinks,breaklinks,
             urlcolor=UniPD,
             linkcolor=UniPD}

\newcommand{\Componenti}{Paolo Eccher \newline Cristian Maschio \newline
Andrea Favero \newline Francesco Parolini}
\newcommand{\Referente}{Francesco Parolini \newline francesco.parolini.1@studenti.unipd.it}
\newcommand{\Gruppo}{Eccher, Maschio, Favero, Parolini}
\newcommand{\Titolo}{Progetto tecnologie web}

\usepackage{lastpage} %info sul # dell'ultima pagina del documento
\usepackage{fancyhdr} %per modificare dimensioni,margini, intestazioni e righe a piè di pagina
\fancypagestyle{plain}{
  % cancella tutti i campi di intestazione e piè di pagina
  \fancyhf{}

  \lfoot{ %piè di pagina
    \Titolo{} \ - \textit{\Gruppo{}}
  }
  \rfoot{Pagina \thepage{} di \pageref{LastPage}} %es: pag: 4 di 10

  %linea orizzontale alle posizioni top e bottom della pagina
  \renewcommand{\headrulewidth}{0pt}  
  \renewcommand{\footrulewidth}{0.3pt}
}
\pagestyle{plain}

%%%%%%%%%%%%%%%%%%%%%%%%%%%%%%%%%%%%%%%%%%%%%%%%%%%%%%%%%%%%%%%%%%%%%%%%%%%%%%%%%%%%%%%%%%%%%%%%%%%%%%%%

\begin{document}

%%%%%%%%%%%%%%%%%%%%%%%%%%%%%%%%%%%%%%%%%%%%%%%%%%%%%%%%%%%%%%%%%%%%%%%%%%%%%%%%%%%%%%%%%%%%%%%%%%%%%%%%
\begin{titlepage}
\centering

\includegraphics[width=50mm]{Images/logo.png}
\vspace*{32px}
{\bigsize \\RELAZIONE\\}
\vspace*{5px}
{\bigsize TECNOLOGIE \\}
\vspace*{5px}
{\bigsize WEB\\}
\vspace*{27px}

\bgroup
\def\arraystretch{1.3}
\centering
\begin{tabular}{c|L{5cm}}
\multicolumn{2}{c}{\textbf{Informazioni sul gruppo} } \\ \hline
	Componenti & \Componenti{} \\
	Referente & \Referente{}
\end{tabular}
\egroup

\vspace*{5px}

\bgroup
\def\arraystretch{1.3}
\centering
\begin{tabular}{c}
\multicolumn{1}{c}{\textbf{Indirizzo web del sito} } \\
	\url{https://www.studenti.math.unipd.it}
\end{tabular}
\egroup

\vspace*{4px}

\begin{tabular}{c|L{3cm}}
\multicolumn{2}{c}{\textbf{Credenziali login admin} } \\ \hline
	Username & admin \\
	Password & admin
\end{tabular}
\quad
\begin{tabular}{c|L{3cm}}
\multicolumn{2}{c}{\textbf{Credenziali login utente} } \\ \hline
	Username & user \\
	Password & user
\end{tabular}

\vspace*{8px}

\includegraphics[width=50mm]{Images/dip_mat.png}\\
\vspace*{\fill} %tutto il resto va in fondo alla pagina
\vspace*{3px}
{\normsize Laurea in Informatica\\ }
\vspace*{0.25px}
{\small Anno Accademico 2017/2018\\ }


\end{titlepage}
%%%%%%%%%%%%%%%%%%%%%%%%%%%%%%%%%%%%%%%%%%%%%%%%%%%%%%%%%%%%%%%%%%%%%%%%%%%%%%%%%%%%%%%%%%%%%%%%%%%%%%%%
\tableofcontents

\section{Abstract}

\section{Target di utenza}

\section{Accessibilità}
	\subsection{Separazione contenuto, presentazione e strutura}
	\subsection{Schema colori}
	\subsection{Tag meta}
	\subsection{Screen Reader}
	\subsection{Facilitazioni per la navigazione}

\section{Usabilità}

\section{Gerarchia dei file}

\section{Struttura}

\section{Presentazione}
	\subsection{Divisione dei file}

\section{Comportamento}

\section{Gestione dei dati}

\section{Validazione e Test}
\end{document}
%%%%%%%%%%%%%%%%%%%%%%%%%%%%%%%%%%%%%%%%%%%%%%%%%%%%%%%%%%%%%%%%%%%%%%%%%%%%%%%%%%%%%%%%%%%%%%%%%%%%%%%%